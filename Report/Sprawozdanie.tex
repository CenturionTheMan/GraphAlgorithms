\documentclass{article}

\usepackage[margin = 0.7in]{geometry}
\usepackage{graphicx}
\usepackage{graphics}
\usepackage[T1]{fontenc}
\usepackage[polish]{babel}
\usepackage[utf8]{inputenc}
\usepackage{float}
\usepackage{tabularx}
\usepackage[table,xcdraw]{xcolor}
\usepackage{lipsum}
\usepackage{titlesec}
\usepackage{minted}
\usepackage{xcolor}

\usepackage{csvsimple}

%TABS
\usepackage[]{booktabs}
\usepackage{tabularray}
\usepackage{multirow}


\title{Zadanie projektowe nr 1}
\author{Michał Dziedziak}
\date{\today}


\titlespacing\section{0pt}{12pt plus 4pt minus 2pt}{0pt plus 2pt minus 2pt}
\titlespacing\subsection{0pt}{12pt plus 4pt minus 2pt}{0pt plus 2pt minus 2pt}
\titlespacing\subsubsection{0pt}{12pt plus 4pt minus 2pt}{0pt plus 2pt minus 2pt}
\setlength{\parskip}{\baselineskip}%
\setlength{\parindent}{0pt}%

\newcommand{\squeezeup}{\vspace{-5mm}}


\begin{document}

\begin{titlepage}
    \begin{center}
        \vspace*{5cm}
        \rule{500pt}{1pt}\\
        \vspace*{0.5cm}
        \LARGE
        \textbf{Zadanie projektowe nr 2}\\
        \Large
        Badanie efektywności algorytmów grafowych w zależności od rozmiaru instancji
        oraz sposobu reprezentacji grafu w pamięci komputera \\
        \vspace*{0.5cm}
        \rule{500pt}{1pt}
    \end{center}

    \vspace*{12cm}

    {\raggedright
        \large
        \textbf{Autor sprawozdania:} Michał Dziedziak,  263901\\
        \textbf{Imię i Nazwisko prowadzącego kurs:} Mgr. inż. Antoni Sterna\\
        \textbf{Dzień i godzina zajęć:} Wtorek, 15:15 - 16:55 TN
    }
\end{titlepage}


\tableofcontents
\listoftables
\listoffigures


\newpage

%CONTENT
\section{Wstęp}
    \subsection{Zarys projektu}
    W ramach projektu w języku C++ zaimplementowane zostały wybrane algorytmy znajdowania najmniejszego
    drzewa rozpinającego i najkrótszej ścieżki w grafie. Do reprezentacji grafów użyta została macierz sąsiedztwa i lista sąsiedztwa. 
    Poniższe sprawozdanie bada zależność czasu od liczby wierzchołków i gęstości grafu podczas wykonywania poszczególnych algorytmów.

    \subsection{Algorytmy}
        \subsubsection{Zaimplemtntowane algorytmy}
            \begin{itemize}
                \item Najmniejsze drzewo rozpinające
                \begin{itemize}
                    \item Algorytm Prima
                    \item Algorytm Kruskala
                \end{itemize}
                \item Najkrótsza ścieżka w grafie
                \begin{itemize}
                    \item Algorytm Dijkstry
                    \item Algorytm Bellmana-Forda
                \end{itemize}
            \end{itemize}

    \subsubsection{Teoretyczne złożoności obliczeniowe}
    W tej sekcji przedstawię teoretyczne złożoności obliczeniowe operacji na grafach, celem późniejszego porównania ich z wynikami testów.
    Złożoności algorytmów różnią się od w zależności od implementacji, poniżej przedstawię oczekiwane złożoności dla wykonanych algorytmów.
    \begin{center}    
        \begin{tabular}[H]{| c | c |}
            \hline
            Algorytm    & Złożoność obliczeniowa wybranej implementacji \\ \hline \hline
            Prima       & $O(ElogV)$    \\ \hline
            Kruskal     & $O(ElogE)$    \\ \hline
            Dijkstra    & $O(ElogV)$    \\ \hline
            Bellman-Ford& $O(V*E)$    \\ \hline
        \end{tabular}
    \end{center}

\section{Plan eksperymentu}
    \subsection{Założenia}
        \begin{itemize}
            \item Macierz sąsiedztwa i niektóre algorytmy wykorzystują umowną wartość maksymalnego int'a do reprezentowania 
            braku krawędzi pomiędzy wierzchołkami lub nieskończonego kosztu dojścia do danego wierzchołka.
            \item Wagi krawędzi są liczbami całkowitymi.
            \item Testy wykonywane są dla grafów posiadających od 50 do 500 wierzchołków (z krokiem równym 50). Dla każdej liczby wierzchołków
            wykonywane są cztery testy dla gęstości grafu równej kolejno 25\%, 50\%, 75\%, 99\%. Test dla każdej kombinacji liczby wierzchołków
            i gęstości grafu wykonywany jest sto razy i ostateczny czas jest średnią tych czasów.
        \end{itemize}

    \subsection{Generowanie grafu}
    Dla każdego testu generowany jest nowy graf. Algorytm analizuje każde potencjalne połączenie pomiędzy wierzchołkami i ustawia je lub nie (ma na to 50\% szans). 
    Kolejno waga takiego połączenia losowana jest z zakresu od 0 lub -1000 (w zależności od algorytmu) do 1000. Algorytm losuje czy dodaje nowe połączenie do momentu,
    kiedy wszystkie pozostałe połączenia muszą zostać stworzone w celu zapewnienia wymaganej gęstości (wtedy tworzy pozostałe wierzchołki z losowymi wagami).
    
    \subsection{Pomiar czasu}
    Do pomiaru czasu została napisana osobna klasa "Timer". Korzysta ona z funkcji QueryPerformanceCounter\\
    i QueryPerformanceFrequency umożliwiających pomiar czasu z dokładnością do mikro sekund. Klasa ta określa upływ czasu, bazując na dokładnym liczniku.
    
    \subsection{Losowanie populacji}
    Do losowania populacji również została napisana osobna klasa RandomGenerator. Umożliwia ona losowanie pojedynczych liczb lub całych zbiorów.
    Korzysta z mt19937 - generatora liczb pseudolosowych.

\section{Wyniki pomiarów}
    
    \subsection{Algorytmy znajdowania minimalnego drzewa rozpinającego}
        
        \subsubsection{Algorytm Prima i macierz sąsiedztwa}
            \begin{table}[H]
                \centering
                \begin{tabular}{|c|c|c|}%
                    \hline
                    \bfseries Liczba wierzchołków & \bfseries Gęstość & \bfseries Czas $[ms]$
                    \csvreader[no head]{Tests/Matrix_Prima.csv}{}% use head of csv as column names
                    {\\\hline\csvcoli&\csvcolii&\csvcoliii}\\
                    \hline
                \end{tabular}
                \caption{Pomiary algorytmu Prima na macierzy sąsiedztwa}
            \end{table}

        \subsubsection{Algorytm Prima i lista sąsiedztwa}
            \begin{table}[H]
                \centering
                \begin{tabular}{|c|c|c|}%
                    \hline
                    \bfseries Liczba wierzchołków & \bfseries Gęstość & \bfseries Czas $[ms]$
                    \csvreader[no head]{Tests/List_Prima.csv}{}% use head of csv as column names
                    {\\\hline\csvcoli&\csvcolii&\csvcoliii}\\
                    \hline
                \end{tabular}
                \caption{Pomiary algorytmu Prima na liście sąsiedztwa}
            \end{table}

        \subsubsection{Algorytm Kruskala i macierz sąsiedztwa}
            \begin{table}[H]
                \centering
                \begin{tabular}{|c|c|c|}%
                    \hline
                    \bfseries Liczba wierzchołków & \bfseries Gęstość & \bfseries Czas $[ms]$
                    \csvreader[no head]{Tests/Matrix_Kruskal.csv}{}% use head of csv as column names
                    {\\\hline\csvcoli&\csvcolii&\csvcoliii}\\
                    \hline
                \end{tabular}
                \caption{Pomiary algorytmu Kruskala na macierzy sąsiedztwa}
            \end{table}
            
        \subsubsection{Algorytm Kruskala i lista sąsiedztwa}
            \begin{table}[H]
                \centering
                \begin{tabular}{|c|c|c|}%
                    \hline
                    \bfseries Liczba wierzchołków & \bfseries Gęstość & \bfseries Czas $[ms]$ 
                    \csvreader[no head]{Tests/List_Kruskal.csv}{} % use head of csv as column names
                    {\\\hline\csvcoli&\csvcolii&\csvcoliii}\\
                    \hline
                \end{tabular}
                \caption{Pomiary algorytmu Kruskala na liście sąsiedztwa}
            \end{table}

    \subsection{Algorytmy znajdowania najkrótszej ścieżki w grafie}
    
    \subsubsection{Algorytm Dijkstry i macierz sąsiedztwa}
        \begin{table}[H]
            \centering
            \begin{tabular}{|c|c|c|}%
                \hline
                \bfseries Liczba wierzchołków & \bfseries Gęstość & \bfseries Czas $[ms]$
                \csvreader[no head]{Tests/Matrix_Dijkstra.csv}{}% use head of csv as column names
                {\\\hline\csvcoli&\csvcolii&\csvcoliii}\\
                \hline
            \end{tabular}
            \caption{Pomiary algorytmu Dijkstry na macierzy sąsiedztwa}
        \end{table}

    \subsubsection{Algorytm Dijkstry i lista sąsiedztwa}
        \begin{table}[H]
            \centering
            \begin{tabular}{|c|c|c|}%
                \hline
                \bfseries Liczba wierzchołków & \bfseries Gęstość & \bfseries Czas $[ms]$
                \csvreader[no head]{Tests/List_Dijkstra.csv}{}% use head of csv as column names
                {\\\hline\csvcoli&\csvcolii&\csvcoliii}\\
                \hline
            \end{tabular}
            \caption{Pomiary algorytmu Dijkstry na liście sąsiedztwa}
        \end{table}

    \subsubsection{Algorytm Bellmana-Forda i macierz sąsiedztwa}
        \begin{table}[H]
            \centering
            \begin{tabular}{|c|c|c|}%
                \hline
                \bfseries Liczba wierzchołków & \bfseries Gęstość & \bfseries Czas $[ms]$
                \csvreader[no head]{Tests/Matrix_BellmanFord.csv}{}% use head of csv as column names
                {\\\hline\csvcoli&\csvcolii&\csvcoliii}\\
                \hline
            \end{tabular}
            \caption{Pomiary algorytmu Bellmana-Forda na macierzy sąsiedztwa}
        \end{table}
    
    \subsubsection{Algorytm Bellmana-Forda i lista sąsiedztwa}
        \begin{table}[H]
            \centering
            \begin{tabular}{|c|c|c|}%
                \hline
                \bfseries Liczba wierzchołków & \bfseries Gęstość & \bfseries Czas $[ms]$
                \csvreader[no head]{Tests/List_BellmanFord.csv}{}% use head of csv as column names
                {\\\hline\csvcoli&\csvcolii&\csvcoliii}\\
                \hline
            \end{tabular}
            \caption{Pomiary algorytmu Bellmana-Forda na liście sąsiedztwa}
        \end{table}

\section{Wykresy}

        \subsection{Algorytmy znajdowania minimalnego drzewa rozpinającego}
        
            \begin{figure}[H]
                \centering
                \includegraphics[scale = 0.85]{Wykresy/mstMatrix.png}
                \caption{Wykres czasu od liczby wierzchołków dla algorytmów wyznaczania minimalnego drzewa rozpinającego i macierzy sąsiedztwa}
            \end{figure}

            \begin{figure}[H]
                \centering
                \includegraphics[scale = 0.85]{Wykresy/mstList.png}
                \caption{Wykres czasu od liczby wierzchołków dla algorytmów wyznaczania minimalnego drzewa rozpinającego i listy sąsiedztwa}
            \end{figure}

            \begin{figure}[H]
                \centering
                \includegraphics[scale = 0.85]{Wykresy/mst25.png}
                \caption{Wykres czasu od liczby wierzchołków dla algorytmów wyznaczania minimalnego drzewa rozpinającego i gęstości równej 25\%}
            \end{figure}

            \begin{figure}[H]
                \centering
                \includegraphics[scale = 0.85]{Wykresy/mst50.png}
                \caption{Wykres czasu od liczby wierzchołków dla algorytmów wyznaczania minimalnego drzewa rozpinającego i gęstości równej 50\%}
            \end{figure}

            \begin{figure}[H]
                \centering
                \includegraphics[scale = 0.85]{Wykresy/mst75.png}
                \caption{Wykres czasu od liczby wierzchołków dla algorytmów wyznaczania minimalnego drzewa rozpinającego i gęstości równej 75\%}
            \end{figure}

            \begin{figure}[H]
                \centering
                \includegraphics[scale = 0.85]{Wykresy/mst99.png}
                \caption{Wykres czasu od liczby wierzchołków dla algorytmów wyznaczania minimalnego drzewa rozpinającego i gęstości równej 99\%}
            \end{figure}

        \subsection{Algorytmy znajdowania najkrótszej ścieżki w grafie}

            \begin{figure}[H]
                \centering
                \includegraphics[scale = 0.85]{Wykresy/gspMatrix.png}
                \caption{Wykres czasu od liczby wierzchołków dla algorytmów wyznaczania najkrótszej ścieżki i macierzy sąsiedztwa}
            \end{figure}

            \begin{figure}[H]
                \centering
                \includegraphics[scale = 0.85]{Wykresy/gspList.png}
                \caption{Wykres czasu od liczby wierzchołków dla algorytmów wyznaczania najkrótszej ścieżki i listy sąsiedztwa}
            \end{figure}

            \begin{figure}[H]
                \centering
                \includegraphics[scale = 0.85]{Wykresy/dijkstraMatrix.png}
                \caption{Wykres czasu od liczby wierzchołków dla algorytmu Dijkstry na macierzy sąsiedztwa}
            \end{figure}

            \begin{figure}[H]
                \centering
                \includegraphics[scale = 0.85]{Wykresy/dijkstraList.png}
                \caption{Wykres czasu od liczby wierzchołków dla algorytmu Dijkstry na liście sąsiedztwa}
            \end{figure}

            \begin{figure}[H]
                \centering
                \includegraphics[scale = 0.85]{Wykresy/gsp25.png}
                \caption{Wykres czasu od liczby wierzchołków dla algorytmów wyznaczania najkrótszej ścieżki i gęstości równej 25\%}
            \end{figure}

            \begin{figure}[H]
                \centering
                \includegraphics[scale = 0.85]{Wykresy/gsp50.png}
                \caption{Wykres czasu od liczby wierzchołków dla algorytmów wyznaczania najkrótszej ścieżki i gęstości równej 50\%}
            \end{figure}

            \begin{figure}[H]
                \centering
                \includegraphics[scale = 0.85]{Wykresy/gsp75.png}
                \caption{Wykres czasu od liczby wierzchołków dla algorytmów wyznaczania najkrótszej ścieżki i gęstości równej 75\%}
            \end{figure}

            \begin{figure}[H]
                \centering
                \includegraphics[scale = 0.85]{Wykresy/gsp99.png}
                \caption{Wykres czasu od liczby wierzchołków dla algorytmów wyznaczania najkrótszej ścieżki i gęstości równej 99\%}
            \end{figure}

\section{Wnioski}
    % \subsection{Przykładowe wyliczenia}
    %     W tej sekcji porównane zostaną czasy otrzymane eksperymentalnie z tymi wynikającymi z teorii. Porównywane będą zależności
    %     czasu wykonania danych operacji dla rozmiarów:


    % \subsection{Komentarz}
    \subsection{Minimalne drzewo rozpinające}
        Algorytm Prima i Kruskala służą do znajdowania minimalnego drzewa rozpinającego w grafie. 
        Czas wykonania obu algorytmów wzrasta wraz z liczbą wierzchołków i gęstością grafu.

        Algorytm Prima polega na wyborze krawędzi o najmniejszej wadze, które nie tworzą cyklu, 
        a następnie dołączaniu ich do drzewa rozpinającego, aż wszystkie wierzchołki zostaną połączone.
        Testy wykazały, że sposób reprezentacji grafu nie wpływa na efektywność algorytmu. Algorytm Prima osiąga lepsze czasy niż algorytm Kruskala zaimplementowany na 
        liście, ale gorsze od jego implementacji macierzowej.

        Algorytm Kruskala natomiast polega na sortowaniu krawędzi według wagi i stopniowym dodawaniu ich do drzewa rozpinającego, jeżeli nie tworzą cyklu. 
        Bazując na pomiarach, ten algorytm działa efektywniej, gdy graf jest reprezentowany w postaci macierzy sąsiedztwa.
        Czas wykonania algorytmu korzystającego z listy sąsiedztwa znacząco rośnie wraz ze wzrostem gęstości grafu.

        Doświadczalnie otrzymane złożoności czasowe algorytmów pokrywają się z tymi teoretycznymi.

    \subsection{Najkrótsza ścieżka w grafie}

        Algorytmy Dijkstry i Bellmana-Forda służą do znajdowania najkrótszej ścieżki między dwoma wierzchołkami w grafie. 
        Podobnie jak w przypadku minimalnego drzewa rozpinającego, czas wykonania obu algorytmów wzrasta z liczbą wierzchołków i gęstością grafu.

        Implementacje algorytmów znajdowania najkrótszej ścieżki, które wykorzystują listę sąsiedztwa, wykazują mocną zależność między czasem wykonania algorytmu
        a gęstością grafu. Z kolei implementacje korzystające z macierzy sąsiedztwa osiągają względnie podobne czasy niezależnie od gęstości grafu. Jest to 
        prawdopodobnie spowodowane tym, że w macierzy sąsiedztwa zawsze sprawdzamy $($ ilość wierzchołków $)^2$ pozycji niezależnie od tego, czy dane połączenie istnieje, czy nie.

        Czas wykonania algorytmu Bellmana-Forda jest znacząco większy niż algorytmu Dijkstry, a dodatkowo rośnie w sposób wykładniczy wraz z liczbą wierzchołków. 
        Dla niższych gęstości grafów, algorytm Bellmana-Forda działa efektywniej na liście sąsiedztwa, natomiast wraz ze wzrostem gęstości, 
        macierz sąsiedztwa staje się bardziej efektywna.

        Podobnie jak w przypadku minimalnego drzewa rozpinającego, doświadczalnie otrzymane złożoności czasowe algorytmów pokrywają się z tymi teoretycznymi.
\end{document}
